\فصل‌باراهنمایی{نظریه‌ی مجموعه‌ها و منطق}
\شروع{سوال}
نشان دهید مجموعه‌ی اعداد گویا مجموعه‌ای شمارا است.
\پایان{سوال}
\راهنمایی{ابتدا نشان دهید
$\IZ^2$
شمارا است و سپس به طور پوشا این مجموعه را به روی
$\IQ$
بناشید}

\شروع{سوال}
فرض کنید $\Sigma$ مجموعه‌ای متناهی باشد.
\شروع{شمارش}
\فقره اگر $\CA$ مجموعه‌ی همه‌ی کلمات با طول \موکد{متناهی} در $\Sigma$ باشد، نشان دهید $\CA$ شمارا است.
\فقره اگر $\CB$ مجموعه‌ی همه‌ی کلمات با طول \موکد{نامتناهی} باشد، نشان دهید $\CB$ ناشمارا است.
\پایان{شمارش}
\پایان{سوال}
\راهنمایی{}

\شروع{سوال}
به عدد مختلط $z$ \موکد{جبری} گفته می‌شود اگر ریشه‌ی چندجمله‌ای $p(x)$ با ضرایب گویا باشد. نشان دهید مجموعه‌ی اعداد جبری شمارا است.
\پایان{سوال}
\راهنمایی{}

\شروع{سوال}
\شروع{شمارش}
\فقره
فرض کنید
$f:\IR\to\IR$
تابعی صعودی باشد. اگر $\CD$ مجموعه‌ی نقاط ناپیوستگی $f$ باشد، نشان دهید $\CD$ مجموعه‌ای شمارا است.
\فقره
فرض کنید $\CF$ مجموعه‌ی توابع صعودی از $\IR$ به $\IR$ باشد. نشان دهید $\CF$ هم‌اندازه‌ی $\IR$ است.
\پایان{شمارش}
\پایان{سوال}
\راهنمایی{}

\شروع{سوال}
به زیرمجموعه‌ی $\CO$ از $\IR$ \موکد{باز} گفته می‌شود هرگاه برای هر $x\in\CO$، بازه‌ای مانند
$(x-\varepsilon,x+\varepsilon)$
وجود داشته باشد ($\varepsilon>0$) به طوری که به تمامی در $\CO$ قرار داشته باشد. نشان دهید هر مجموعه‌ی باز را می‌توان به صورت اجتماع (نه لزوما مجزای) حداکثر شمارا بازه نوشت.
\پایان{سوال}
\راهنمایی{}

\شروع{سوال}
فرض کنید $\CF$ مجموعه‌ی توابع اکیدا صعودی از $\IN$ به $\IN$ باشد. نشان دهید $\CF$ هم‌اندازه‌ی $\IR$ است.
\پایان{سوال}
\راهنمایی{}

%\شروع{سوال}
%\پایان{سوال}
%\راهنمایی{}